\documentclass[a4paper,10pt]{article}

% IMPORTS
\usepackage{amsfonts}
\usepackage{amsmath}
\usepackage{amssymb}
\usepackage{graphicx}
\usepackage{titlesec}
\usepackage{wrapfig}
\usepackage[ngerman]{babel}
\usepackage[utf8x]{inputenc}
\usepackage{pdfpages}
\usepackage{placeins}
\usepackage{color}
\usepackage{eurosym}
\usepackage{xargs}
\usepackage{xcolor}
\usepackage{subcaption}
\usepackage{hyperref}
\usepackage{cleveref}
\usepackage{stix}
\usepackage[margin=2cm]{geometry}
\usepackage[colorinlistoftodos,prependcaption,textsize=tiny]{todonotes}

% CONFIG
\clubpenalty = 9000
\widowpenalty = 9000
\displaywidowpenalty = 9000
\titlespacing\section{0pt}{14pt plus 4pt minus 2pt}{2pt plus 2pt minus 1pt}
\titlespacing\subsection{0pt}{10pt plus 4pt minus 2pt}{2pt plus 2pt minus 1pt}
\setlength{\parindent}{0pt}
\setlength{\parskip}{0.5em}
\setcounter{tocdepth}{2}

% -----------------------------------------------------------------------------
\begin{document}

% \section{Graphentheorie}

\subsection{Partielle Ordnung}

\textbf{Aufgabenstellung:} Gegeben eine Partielle Ordnung durch Menge $M$ und Relation $R\subseteq M\times M$.
\begin{itemize}
    \item Zeichne den zugehörigen Graphen.
    \item Bestimme maximale Ketten und Antiketten.
\end{itemize}





% \section{Klassische AbhAna (gen, kill, in, out) (Übung 3)}

\section{Diophantische GLS lösen (Übung 4)}
\begin{center}
\begin{tabular}{rc|c|c}
\textbf{Gesucht:}                      & $x$   & $A * x = c$              & $\hrectangle * \vrectangleblack = \vrectangleblack$ \\
\hline
\textbf{Umformung Zeilenstufenform:}   & $S$   & $U * A^T = S$            & $\square * \vrectangle = \vrectangle$\\
                                       &       & $(A^T|I) \leadsto (S|U)$ & $(\vrectangle | \square)$ \\
\hline
\textbf{Ganzzahlige Lösungsparameter:} & $t$   & $t * S = c^T$            & $\hrectangleblack * \vrectangle = \hrectangleblack$ \\
\hline
\textbf{Lösung:}                       & $x$   & $t * U = x^T$              & $\hrectangleblack * \square = \hrectangleblack$ \\
\end{tabular}
\end{center}
%!TEX root = Rezepte.tex

\def\AT{%
\begin{array}{ccc}
-2 & -2 & -8 \\
 1 &  4 & 10 \\
 0 &  2 &  4 \\
 1 &  2 &  6
\end{array}
}%
\def\I{%
\begin{array}{cccc}
 1 &  0 &  0 &  0 \\
 0 &  1 &  0 &  0 \\
 0 &  0 &  1 &  0 \\
 0 &  0 &  0 &  1
\end{array}
}%
\def\S{%
\begin{array}{ccc}
 1 &  0 &  2 \\
 0 &  2 &  4 \\
 0 &  0 &  0 \\
 0 &  0 &  0
\end{array}
}%
\def\U{%
\begin{array}{cccc}
  0 &  0 & -1 & -1 \\
  0 &  0 &  1 &  0 \\
  1 &  0 & -1 &  2 \\
  0 &  1 & -1 & -1
\end{array}
}%

\[
\left(\vphantom{\AT}\right.\kern-2\nulldelimiterspace
\overbrace{\AT}^{A^T}\kern-\nulldelimiterspace
\left.\middle|\vphantom{\AT}\right.
\overbrace{\I}^{I}\kern-\nulldelimiterspace
\left.\vphantom{\AT}\right)
\leadsto
\left(\vphantom{\S}\right.\kern-2\nulldelimiterspace
\overbrace{\S}^{S}\kern-\nulldelimiterspace
\left.\middle|\vphantom{\S}\right.
\overbrace{\U}^{U}\kern-\nulldelimiterspace
\left.\vphantom{\S}\right)
\]

\[
\overbrace{
(t_1, t_2, t_3, t_4)
}^t
*
\left(\vphantom{\S}\right.\kern-2\nulldelimiterspace
\overbrace{\S}^{S}\kern-\nulldelimiterspace
\left.\vphantom{\S}\right)
=
\overbrace{
(1, -4, -6)
}^{c^T}
\quad \Rightarrow \quad
\left\lbrace
\begin{array}{rcl}
t_1 &=& \hphantom{-}1 \\
2t_2 &=& -4 \\
2t_1 + 4t_2 &=& -6\\
t_3, t_4 &\in& \mathbb{Z}
\end{array}
\right.
\]

\[
\overbrace{
(1, -2, t_3, t_4)
}^t
*
\left(\vphantom{\U}\right.\kern-2\nulldelimiterspace
\overbrace{\U}^{U}\kern-\nulldelimiterspace
\left.\vphantom{\S}\right)
=
\overbrace{
(t_3, t_4, -t_3-t_4-3, 2t_3-t_4-1)
}^{x^T}
\quad \Rightarrow \quad
x=
\left(
\begin{array}{rcl}
x_1 &=& t_3 \\
x_2 &=& t_4 \\
x_3 &=& -t_3-t_4-3 \\
x_4 &=& 2t_3-t_4-1 \\
\end{array}
\right)
\]

\section{Fourier-Motzkin (Übung 4)}
Ordne Variablen aufsteigend nach Anzahl der Gleichungen (äußerste Schleifenvariable zuletzt),
anschließend sukzessive Elimination der einzelnen Variablen: sei $x_j$ die zu eliminierende Variable
\begin{enumerate}
	\item \label{itm:fm-aufl} Löse die Gleichungen nach $x_j$ auf\\
		$\Rightarrow$ Ergebnis: untere Schranken für $x_j$, obere Schranken für $x_j$ und Ungleichungen, die $x_j$ nicht beinhalten
	\item \label{itm:fm-elim} Lösche die $x_j$ beschränkenden Ungleichungen und füge stattdessen die Paare aller möglichen Ungleichungskombinationen ein, die sich ergibt, wenn man alle Unterschranken von $x_j$ allen Oberschranken von $x_j$ gegenüberstellt
	\item Weiter mit $x_{j+1}$ und den Ungleichungen aus \Cref{itm:fm-aufl}, die $x_j$ nicht beinhalten, und den kombinierten Ungleichungen aus \Cref{itm:fm-elim}.
\end{enumerate}
\textbf{erfolgreiche Termination:} keine Ungleichung oder keine Variable geblieben\\
\textbf{erfolglose Termination:} die letzte Ungleichung (ohne Variable) ist nicht erfüllt

\section{Banerjee (Übung 6)}
\section{Feautrier (Übung 7)}
\section{Single Assignment (Übung 7)}
\section{Schnelltest/Vereinfachungen (GCD- und Extreme Value Test) (Übung 8)}
\section{CfFada (Übung 9)}
\section{Omega (Übung 10)}

\end{document}
