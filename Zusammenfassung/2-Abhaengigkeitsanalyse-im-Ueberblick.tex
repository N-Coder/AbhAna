%!TEX root = Zusammenfassung.tex

\section{Abhängigkeitsanalyse im Überblick}
\subsection{Programm-Modell}

\subsubsection{Datentypen}
\begin{enumerate}
  \item Als Datentypen erlauben wir nur (mehrdimensionale) Arrays mit beliebigen Einträgen.
    Beachte: Skalare sind 0-dimensionale Arrays.
  \item Lediglich Indizes von Schleifen werden nicht als 0-dimensionale Arrays, sondern wirklich als Integers betrachtet.
  \item Darüberhinaus gibt es symbolische Konstanten vom Typ Integer, die oft auch als Strukturparameter bezeichnet werden.
\end{enumerate}

\subsubsection{Kontrollstrukturen}
\begin{enumerate}
  \item Das Basiselement ist die Zuweisung an ein Arrayelement.
  \item Die einzige Kontrollstruktur ist die Schleife (beliebig verschachtelte Anwendung willkommen).
  \item Als Schleifenunter- bzw. -obergrenzen sind im originalen Programm-Modell nur Maxima bzw. Minima affiner Ausdrücke in umgebenden Schleifenindizes und in Strukturparametern erlaubt; das wird auch der Schwerpunkt der Vorlesung sein, wenngleich diese Restriktionen mittlerweile theoretisch aufgelöst wurden.
\end{enumerate}

\subsubsection{Iterationen}
  \begin{enumerate}
  \item Durch die Schleifen entstehen verschiedene Instanzen von Zuweisungen (Operations).
  \item Identifiziert werden Operations durch Angabe des Statements und des Iterationsvektors (Vektor, der sich aus den Werten der Indizes der umgebenden Schleifenindizes zusammensetzt).
  \item Der Raum aller Operations heißt Index- oder Iterationsraum aller Statements.
    (Manchmal werden wir die Indexräume von verschiedenen Statements auch übereinanderlegen und somit nur einen Indexraum für mehrere Statements erhalten,
    falls die Statements dieselben umgebenden Schleifen besitzen.)
\end{enumerate}

\subsubsection{Ausführungsreihenfolge}
\begin{enumerate}
  \item Für sequentielle Schleifen gilt: In einem perfekt
    verschachtelten Schleifensatz (die Zuweisungen stehen alle in
    demselben Rumpf der innersten Schleife) werden verschiedene
    Instanzen des Schleifenrumpfes gemäß der lexikographischen Ordnung
    der Iterationsvektoren abgearbeitet.
  \item Operation $a$ ist gemäß der sequentiellen Ausführungsordnung
    kleiner als (also vor) Operation $b$, i.Z. $a \prec b$ gdw. ihre
    Iterationsvektoren, eingeschränkt auf gemeinsam umgebende Schleifen,
    lexikographisch kleiner sind, oder diese Vektoren gleich sind und
    $a$ im Programmtext vor $b$ steht.
    %wenn nach dem Entfernen aller gemeinsam umgebenden (äußeren) Schleifen
\end{enumerate}

\subsubsection{Abhängigkeiten}
\begin{enumerate}
  \item geben an, für welche Operations die Ausführungsreihenfolge zwingend vorgeschrieben ist. Formale Definition: später.
\end{enumerate}

\subsubsection{Abhängigkeitsgraphen}
\begin{enumerate}
  \item Iterationsabhängigkeitsgraph: je Operation (oder für jeweils übereinandergelegte Operationen) ein Knoten mit Abhängigkeiten als Kanten; sehr genau, aber u.U. unbeschränkt.
  \item Statementabhängigkeitsgraph: nur je Statement ein Knoten; kleiner, aber ungenauer.
\end{enumerate}

\subsubsection{Parallelisierung}
\begin{enumerate}
  \item wirklich notwendige Ordnung von den Abhängigkeiten vorgeschrieben
  \item paralleles Programm (Herleitung später)
\end{enumerate}



\subsection{Abhängigkeiten im allgemeinen}
Abhängigkeiten in einem Programm spiegeln die durch das Programm
zwingend vorgeschriebene Ausführungsreihenfolge von \emph{Instanzen von
Anweisungen} wider. In imperativen Programmen erfolgt ihre Berechnung
üblicherweise durch die Ermittlung von Speicherstellen, auf die mehrfach
zugegriffen wird.

\subsubsection{Bedingungen}
Eine Operation $o_t$ ist abhängig von einer Operation $o_s$, gdw. folgende Bedingungen erfüllt sind:
\begin{description}
\item[Speicherkonflikt] ~\\$o_s$ und $o_t$ müssen auf dieselbe Speicherzelle zugreifen.
\item[Existenz] ~\\$o_s$ und $o_t$ müssen beide durch das Programm wirklich generiert werden.
\item[Ordnung] ~\\$o_s$ wird gem. der sequentiellen Ausführungsreihenfolge vor $o_t$ ausgeführt---es kann nur eine später ausgeführte Instanz von einer früher ausgeführten Instanz abhängig sein.
\item[Optimierung] ~\\Zwischen den zwei Instanzen wird der Wert der Speicherzelle nicht mehr modifiziert, d.h., es besteht eine ``direkte Abhängigkeit'.
\item[Bernsteinzusatz] ~\\Mindestens einer der in Konflikt stehenden Zugriffe muß schreibend sein.
\end{description}

\paragraph{Bemerkungen:} Die Optimierung wird bei der praktischen Berechnung oft vernachlässigt.

Die Bernstein-Bedingungen werden i.a. explizit angegeben (3 Fälle) und
beinhalten Ordnung, Existenz und Speicherkonflikt implizit. Je nach
Analyse-Methode wird die Zusatzbedingung von Bernstein implizit bei der
Auswahl der zu betrachtenden Zugriffspaare berücksichtigt.


\subsection{Typen von Abhängigkeiten}
Je nachdem, ob das Ziel oder die Quelle ein lesender oder ein schreibender Zugriff ist, werden verschiedene \emph{Typen} von Abhängigkeiten unterschieden:

\begin{center}
\begin{tabular}{||r|cc||}
\hline
Ziel & \multicolumn{2}{|c||}{Quelle}\\
 & liest & schreibt \\
\hline
liest & input & true\\
schreibt & anti & output\\
\hline
\end{tabular}
\end{center}

\paragraph{Bemerkungen:}
\begin{enumerate}
\item i.A. werden input dependences wegen der Bernstein-Zusatzbedinung
  vernachlässigt.
\item true dependences, die das Optimierungskriterium erfüllen, heißen auch flow dependences.
  (ACHTUNG: Diese Begriffe werden in der Literatur uneinheitlich verwendet.)
\end{enumerate}

\paragraph{Graphische Darstellung:}
Die graphische Darstellung erfolgt entweder durch den
\emph{Iterations-Abhängigkeitsgraphen} oder den
\emph{Statement-Abhängigkeitsgraphen}.
