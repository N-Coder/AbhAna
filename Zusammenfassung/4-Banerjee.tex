%!TEX root = Zusammenfassung.tex
\section{Abhängigkeitsanalyse nach Banerjee}


\subsection{Normalisierung des Schleifenkopfes}
Die Behandlung von Schrittweiten ungleich von eins, insbesondere von
negativen Schrittweiten, macht die Abhängigkeitsanalyse aufwendig. Daher
normalisiert man Schleifen oft wie folgt:


  % {\sf for $i:=x$ to $y$ step $s$ do $R(i)$}

\begin{minipage}{.4\textwidth}
  \begin{algorithm}[H]
      \SetKw{KwStep}{step}
      \For{$i:=x$ \KwTo $y$ \KwStep s}{
        \( R(i) \)
      }
  \end{algorithm}
\end{minipage}
\begin{minipage}{.5\textwidth}
    \qquad $\to$ \qquad
    \begin{algorithm}[H]
          \SetKw{KwStep}{step}
          \For{\( i' := 0\) \KwTo \( \frac{y\!-\!x}{s} \) \KwStep 1}{
            \( R(i' \cdot s + x) \)
          }
      \end{algorithm}
\end{minipage}
  % {\sf for $i':=0$ to ${y\!-\!x}\over s$ step $1$ do $R(i'*s+x)$}
  ~\\
(vgl. \cite{Zima90}, S. 175)


\subsection{Konflikt-Gleichungssystem}
Zwei Operations $o_s$ und $o_t$ müssen auf dieselbe Speicherzelle
zugreifen, d.h., die Variablennamen und die Array-Indizes, die $o_s$ und
$o_t$ ansprechen, müssen in allen Komponenten übereinstimmen. Für die
entsprechenden Iterationsvektoren $i_s$ und $i_t$ muß also in
Matrixnotation gelten:
$$i_s * A + a_0 = i_t * B + b_0,$$ wobei jede Spalte von $A$ ($B$) die
Koeffizienten eines Array-indexes beinhaltet und $a_0$ ($b_0$) den
konstanten Anteil bechreibt \cite{Ban93}.

Anders notiert: $$(i_s, i_t) * \left( {A \atop -B} \right) = b_0-a_0$$

Lösung wie im Abschnitt für mathematische Grundlagen
beschrieben.


\subsection{Existenz-Ungleichungssystem}
Die zwei Operations $o_s$ und $o_t$ müssen beide durch das Programm
wirklich generiert werden, d.h., die Iterationsvektoren $i_s$ und $i_t$
müssen im jeweiligen Indexraum der entsprechenden Statements
sein. Matrixnotation: $$p_0 \leq i * P \hbox{\qquad bzw. \qquad} i*Q \leq
q_0$$ für die unteren bzw. oberen Grenzen der Indexräume. Diese
Ungleichungen müssen natürlich sowohl für $i_s$ als auch für $i_t$
aufgestellt werden.

Die -- ggfs. parametrisierten -- Lösungen des Konflikt-Gleichungssystems
kann man in das Ungleichungssystem einsetzen und so die Lösbarkeit
innerhalb des Indexraums überprüfen, bzw. ggfs. die Wahl der Parameter
einschränken.


\subsection{Ordnungsrelation und Abhängigkeitstypen}
Bis hierher haben wir zwei Operations $o_1$ und $o_2$, die eine
Abhängigkeit verursachen. Frage: Wer ist Quelle, wer Ziel?  Wir wissen:
Operation $o_s$ muß gem.  der sequentiellen Ausführungsreihenfolge vor
$o_t$ ausgeführt werden. Lösung: teile die (ggfs. parametrisierten)
Paare von Iterationsvektoren so in zwei Mengen ein, daß in einer stets
gilt $o_1 \prec o_2$ und in der anderen $o_2 \prec o_1$ (ggfs. unter
Aufteilung der Räume für die Wahl der Parameter). Dadurch sind jeweils
Quelle und Ziel identifiziert, was damit auch die Typen der
Abhängigkeiten lt. Tabelle im vorigen Abschnitt festlegt. (Die Ordnung ist bei dem Verfahren nach Banerjee nicht a priori bekannt.)


\subsection{Optimierung}
Die Optimierung wird bei Banerjee komplett vernachlässigt.


\subsection{Randbemerkung: Sonderfälle}
Banerjee stellt Sonderfälle vor, in denen sich die Analyse etwas vereinfacht.
\begin{enumerate}
\item ``regulärer'' Schleifensatz: perfekt, $P=Q$ und $A=B$, dann für $d
  = i_s-i_t$. Dadurch $$ d * A = a_0-b_0 \hbox{\qquad und \qquad} p_0-q_0 \leq
  d*P \leq q_0-p_0$$. Die Halbierung der Variablenzahl macht das
  Gleichungssystem einfacher und die Ordnung handlicher.
\item ``rechteckiger'' Schleifensatz: $P=Q=I_m$ (Identität) und $A$ und
  $B$ in Diagonalform (aber nicht unbeding identisch). Dadurch
  unabhängige $m$ Gleichungen mit je 2 Variablen, die unabhängig
  voneinander aufgelöst werden können. Die Zeilenstufenreduktion wird
  daher etwas übersichtlicher, aber im Endeffekt keine spürbare
  Erleichterung.
\end{enumerate}


\subsection{Beispiel zur Abhängigkeitsanalyse nach Banerjee}
\textit{Programmcode:}\\

\begin{procedure}
\SetAlgoLined
\For{$i=10$ \KwTo $200$}{
    \For{$j=7$ \KwTo $167$}{

        \textbf{S:} $X[2i+3, 5j-1, j]$ :=...\\
        \textbf{T:} ... := ... $X[i-1,2i-6,3j+2]$
    }
}
\end{procedure}
~\\
\textit{Konfliktgleichungssystem:} (gestrichene Var. f. T)\\

\begin{align*}
2i+3 &= i^\prime  - 1\\
5j-1 &= 2i^\prime - 6\\
 j   &= 3j^\prime + 2
\end{align*}

nach Banerjee:\\

$A=
\left(
\begin{array}{ccc}
2 & 0 & 0\\
0 & 5 & 1
\end{array}
\right), a_0 = (3,-1,0)$

$B=
\left(
\begin{array}{ccc}
1 & 2 & 0\\
0 & 0 & 3
\end{array}
\right), b_0 = (-1, -6, 2)$


$(i,j) * A + a_0 = (i^\prime, j^\prime) * B + b_0 \\
\Leftrightarrow
(i,j,i^\prime,j^\prime) *
\left(
\begin{array}{c}
A\\
-B
\end{array}
\right)
= b_0 - a_0$\\
Eingesetzt:\\
$(i,j,i^\prime,j^\prime) *
\left(
\begin{array}{ccc}
2 & 0 & 0\\
0 & 5 & 1\\
-1&-2 & 0\\
0 & 0 &-3
\end{array}
\right)
= (-4,-5,2)$

\textit{Lösen}

$\left(
\begin{array}{ccc|cccc}
 2 & 0 & 0 & 1 & 0 & 0 & 0 \\
 0 & 5 & 1 & 0 & 1 & 0 & 0 \\
-1 &-2 & 0 & 0 & 0 & 1 & 0 \\
 0 & 0 &-3 & 0 & 0 & 0 & 1
\end{array}
\right) \leadsto \left(
\begin{array}{ccc|cccc}
-1 &-2 & 0 & 0 & 0 & 1 & 0 \\
 0 & 1 & 1 & 1 & 1 & 2 & 0 \\
 0 & 0 & 1 & 5 & 4 &10 & 1 \\
 0 & 0 & 0 &15 &12 &30 & 4
\end{array}
\right)$


$(t_1,t_2,t_3,t_4) * S = (-4, -5, 2) \\
\Rightarrow t_1 = 4; t_2= 3; t_3 = -1; t_4 \in \mathbb{Z} \\
\Rightarrow (i,j,i^\prime,j^\prime) = (t_1,t_2,t_3,t_4) * U =
(-2+15t_4; -1+12t_4; 30t_4; -1+4t_4)$
somit kann man folgende Abhängigkeiten folgern:\\
$\langle (-2 + 15t_4; -1+12t_4), S \rangle \stackrel{?}{\leftrightarrow}
 \langle (30t_4;-1+4t_4), T \rangle$ für $t_4 \in \mathbb{Z}$
~\\
\textit{Existenzungleichungssystem:}\\

$P = Q =
\left(
\begin{array}{cc}
1 & 0 \\
0 & 1
\end{array}
\right), p_0 = (10,7), q_0 = (200,167)$
~\\
\textit{Gleichungen:}\\

$\begin{array}{cc}
\underbrace{
\begin{array}{ccc}
10 &\leq -2+15t_4 &\leq 200 \\
7  &\leq -1+12t_4 &\leq 167
\end{array}
}_{\text{Existenz von} (i,j) \text{bei S}} &
\underbrace{
\begin{array}{ccc}
10 &\leq 30t_4 &\leq 200 \\
 7 &\leq-1+4t_4 &\leq 167
\end{array}
}_{\text{Existenz von} (i,j) \text{in T}}

\end{array}$

$\stackrel{FM}{\leadsto} 2 \leq t_4 \leq 6$
~\\
\textit{Ordnung:} Richtung der Abhängigkeit $\leftarrow$ oder $\rightarrow$ ?\\

Lexikographische Ordnung: $i < i^\prime$ oder $i = i^\prime \land j < j^\prime$\\
\begin{equation}
\begin{split}
i<i^\prime &\Leftrightarrow -2+15t_4 < 30t_4\\
           &\Leftrightarrow    15t_4 > -2\\
           &\Leftrightarrow      t_4 >-\frac{2}{15}
\end{split}
\end{equation}
da $t_4 \geq 2 \Leftrightarrow true$\\
Lösung $\underbrace{\langle (-2 + 15t_4; -1+12t_4), S \rangle}_{\text{schreibt} \Rightarrow \text{true Dependence}} \rightarrow
 \underbrace{\langle (30t_4;-1+4t_4), T \rangle}_{\text{liest}}$ für $2 \leq t_4 \leq 6$\\

Distanzvektoren: \glqq Ziel-Quelle\grqq\\
$(30t_4; -1+4t_4) - (-2+15t_4; -1+12t_4) =\\
(2+15t_4; -8t_4)$ für $2\leq t_4 \leq 6)$.

Richtungsvektor: $(+,-)$\\
Level: 1\\
~\\
\textit{Statement-Abhängigkeits-Graphen:}\\

\entrymodifiers={++[o][F-]}
\xymatrix{
S \ar[d]^{( 2+15t_4; -8t_4 )} \\
T
}
~\\
\textit{Optimierung:} nicht bei Banerjee.\\
~\\
Somit führt Banerjee zum Abhängigkeitsvektor und gibt alle Abhängigkeiten inklusive einiger eigentlich nutzloser aber für den Algorithmus notwendiger \glqq Initialisierungsabhängigkeiten\grqq\ zurück (Initialisierungsschrott).
