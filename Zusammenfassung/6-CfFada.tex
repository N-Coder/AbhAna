%!TEX root = Zusammenfassung.tex

\section{Verfahren via CfFada}
CfFada heißt ausgeschrieben \glqq Controllflow based Fuzzy array dataflow\grqq.\\
Dieses weiteres Verfahren wird motiviert durch die Existenz von \glqq if-Statements\grqq\ , \glqq unstrukturierte Programme\grqq\ und die erhöhte \glqq Präzision\grqq.


\subsection{Idee}
\begin{enumerate}
\item Zusammenführen von
  \begin{enumerate}
  \item traditionellen (universell einsetzbar, ungenau) und
  \item einem Feautrier-ähnlichem Verfahren
  \end{enumerate}
\item (Effizienz)
\end{enumerate}


\subsection{Prinzipielles Vorgehen}
\begin{enumerate}
\item Konstruiere den Kontrollflußgraphen (CFG)
\item Für jeden Zugriff $a$:
  \begin{enumerate}
  \item Annotiere den CFG mit den für a relevanten Relationen - unter Berücksichtigung der umgebenden Prädikate
  \item starte mit leerer Abhängigkeitsmenge (alles bottom) und Ordnungsrelation $(=,...,=)$ bei a. (\( i = i^\prime = \dots \))
  \item durchlaufe den CFG rückwärts
  \item bei Schleifen: durchlaufe die Rückwärtskanten von innen nach außen und passe die aktuell gültige Ordnungsrelation an
  \item bei \glqq conflicting accesses\grqq : mische die neuen und alten Abhängigkeiten unter Berücksichtigung der Optimierung
  \end{enumerate}
\end{enumerate}


\subsection{Algorithmus}
\subsubsection{Zeichnen des Kontrollflussgraphen}
Der Kontrollflussgraph ist ein gerichteter Graph.
\begin{enumerate}
    \item Zeichne einen Knoten für den Programmstart
    \item Zeichne für eine for-Schleife drei Knoten (Kopf, Iteration, Ausstieg) verbunden durch zwei Kanten Kopf $\rightarrow$ Iteration und Kopf $\rightarrow$ Ausstieg
    \item Zeichne für eine if-else-Fallunterscheidung vier Knoten (Bedingung, true-Pfad, false-Pfad, Ende), die rautenförmig verbunden sind (bis auf weitere Satements in den true- und false-Pfaden).
    \item Zeichne für jede Zuweisung $S$ eine passend beschriftete Kante $S$, die in einem neuen Knoten endet
    \item Verbinde aufeinanderfolgende Anweisungen mit Vorwärts-, Schleifenwiederholungen durch Rüchwärtskanten.
        Solche Kanten, die nicht zu einem benannten (Zuweisungs-) Statement wie $S$ oder $T$ gehören werden bei 1 beginnend durchnummeriert. Die Reihenfolge spielt dabei keine Rolle.
    \item Der Pfad zum Programmende (üblicherweise der letzte Schleifenausstieg) erhält einen Endknoten
\end{enumerate}

\subsubsection{Bestimmen der Quellen pro Abhängigkeit}
In diesem Schritt werden die möglichen Quellen \textit{src} pro Leseoperation bestimmt. Zunächst werden die Quellen des Lesevorgangs mit $\bot$ (Bottom, keine Quelle vorhanden / undefinierter Wert) identifiziert und nach und nach aufgefüllt.

Währenddessen wird die \textit{Ordnung} zwischen den Indizes der Leseoperation und denen der momentan betrachteten Schleifeniterationen modifiziert. Die Ordnung $(=,<)$ bedeutet etwa, dass für die Indizes $i=i'\wedge j<j'$ gilt, wobei $i$ und $j$ die Schreiboperation, $i'$ und $j'$ die Leseoperation indzieren.
\begin{enumerate}
    \item Setze $\mathit{src}={\bot}$ und beginne mit der Ordnung $(=,=)$.
    \item Laufe Kante für Kante rückwärts durch den Kontrollflussgraphen, beginnend bei der zu untersuchenden Leseoperation.
        \begin{enumerate}
            \item Bei einer Schreiboperation wird geprüft, ob der Konflikt zur Leseoperation mit der Momentanen Ordnung vereinbar ist. Falls ja, wird die (Konflikt-)Bedingung für die Abhängigkeit ausgewertet und \textit{src} entsprechend angepasst.

                Immer wenn sich \textit{src} ändert, muss eine Mischoperation (Merge) durchgeführt werden, bei dem aus zwei Schreiboperationen auf die gleiche Speicherzelle die spätere (überschreibende) ausgewählt wird.
            \item Bei der Verzweigung an einem Schleifenkopf wird die Rückwärtskante gewählt, falls die Schleife noch nicht behandelt wurde oder sich bei ihrem letzten Durchlauf die Ordnung geändert hat. Ansonsten wird sie über die Eingangskante verlassen.

                Wird eine Schleife verlassen, so wird die Ordnung in der Schleifenvariablen auf $*$ (beliebig) gesetzt.
            \item An einer if-else-Verzweigung werden beide Pfade durchlaufen, die Einträge in \textit{src} mit
                den entsprechenden Prädikaten $P$ versehen und beim Erreichen des Bedingungsknotens ein \textit{Merge} durchgeführt.
            \item Bei der Rückwärtskante einer Schleife wird die Ordnung angepasst: Zählt die Schleife hoch, wechselt die Ordnung in der jeweiligen Variable von $=$ auf $<$.
            \item Leere Kanten werden übersprungen.
        \end{enumerate}
    \item Der Algorithmus terminiert, sobald für kein Indextupel $(i,j)$ mehr $\bot$ als Quelle auftritt.
\end{enumerate}


\subsection{Anmerkungen}
\begin{enumerate}
  \item Implementierung mit Omega (Presburger Arithmetik) \( \rightarrow \) Genauigkeit
  \item unstrukturierte Programme \( \rightarrow \) Abschätzung der Ordnungsrelation
  \item vorzeitiges Analyse-Ende möglich, wenn alle möglichen letzten Quellen gefunden wurden.
  \item Der Hauptunterschied zu Feautrier besteht in der Tatsache, dass CfFada mit if-Statemants umgehen kann.
  \item textuell getragene Abhängigkeiten entspricht sprachlich den Schleifenunabhängigen Abhängigkeiten
  \item Mischen: Schneiden und Optimierung
  \item \( \Box := \) Merge/Mischen
  \item \( W(6) := \) Write-Set von 6
\end{enumerate}


\subsection{Beispiel}
\begin{figure}[h]
    \begin{subfigure}[c]{.5\linewidth}
        \paragraph{Programmcode}~\\
        \begin{procedure}[H]
        \SetAlgoLined
        \For{$i=0$ \KwTo $N$}{
            \For{$j=0$ \KwTo $N$}{
                \textbf{3:} $A[i+j+1]$ = ... \\
                \eIf{$P$}
                    {\textbf{6:} $A[i+j]$ = ...\\}{}
                \textbf{7:} ... = $A[i+j]$
            }
        }
        \end{procedure}
        \caption{Programmcode}
    \end{subfigure}
    \begin{subfigure}[c]{.5\linewidth}
        \includegraphics[width=\linewidth]{images/cffada.png}
        \caption{Ablaufgraph}
    \end{subfigure}
\end{figure}

\paragraph{Kontext}~\\
Read-Statement 7\\
Operationen: $\langle (i,j), 7 \rangle$ für $0 \leq i, j \leq N$

\paragraph{Annotierung}~\\
Statement 3\\
Operationen: $\langle (i^\prime,j^\prime), 3 \rangle\) für \( \underbrace{0 \leq i^\prime, j^\prime \leq N}_{Existenz} \land \underbrace{i^\prime + j^\prime + 1 = i + j}_{Konflikt}$\\
Statement 6\\
Operationen: \dots\\

\paragraph{Durchlauf}~\\
\begin{enumerate}
    \item {Ordnung: \(i=i^\prime \land j=j^\prime\) \\
        \(src^{(0)} = {\perp}\)\\
        \begin{enumerate}
            \item Start: 7
            \item die 4 hoch: (Kante mit Label 4)\\
                  Keine neuen Quellen\\
                  \(scr_{4}^{(1)} = \lbrace \perp \rbrace\)
            \item die 6 hoch: (überprüfe Existenzun- und Konfliktgleichungen)\\
                  \(src_{6}^{(1)} = src^{(0)} \Box W(6) = \)
    \(\begin{cases}  \langle (i,j), 6 \rangle , & \mbox{ für } P^\prime(i,j) \\
                     \perp ,                    & \mbox{ sonst }
    \end{cases}  \)

            \item bei n1 Mischen:\\
                \(scr_{n1}^{(1)} =
    \begin{cases} \langle (i,j),6 \rangle, & \mbox{ für } p^\prime(i,j) \\
                  \perp,                   & \mbox{ sonst }
                \end{cases} \)

            \item die 3 hoch:\\
                  Keine neuen Quellen, da die aktuelle Ordnung $\lightning$ Konfliktgleichung. (Wäre P immer true, könnte man an n1 abbrechen.)
            \item die 13 hoch \(\Rightarrow\)

        \end{enumerate}
        }
    \item {Ordnung: \(i=i^\prime \land j^\prime < j\) \\
      \begin{enumerate}
         \item die Kanten 7, 4 bringen nichts neues.
         \item die 6 hoch: \\
           Keine neuen Quellen, da die aktuelle Ordnung $\lightning$ Konfliktgleichung.
         \item n1: nichts neues zu mischen \(\Rightarrow\) alles bleibt
         \item die 3 hoch: \\
           \(src^{(2)} = src^{(1)} \Box W(3) = \) \\ \( =
           \begin{cases} \langle ( i^\prime, j^\prime), 3) \rangle, & \mbox{ für } i^\prime + j^\prime + 1 = i + j \land \neg P^\prime(i,j) \land 0 \leq i^\prime \leq N \land 0 \leq j^\prime \leq N \\ & \mbox{ mit aktueller Ordnung}\\
             \langle (i,j),6 \rangle , & \mbox{ für } P^\prime(i,j) \\
             \perp, & \mbox{sonst}
           \end{cases}
\)
           \\ \(\stackrel{j^\prime+1=j}{=}
           \begin{cases}
             \langle ( i, j-1), 3 \rangle, & \mbox{ für }\neg P^\prime(i,j) \land j > 0 \land 0 \leq j-1 \leq N \\
             \langle (i,j),6 \rangle , & \mbox{ für } P (i,j) \\
             \perp , & \mbox{ sonst, d.\,h. } j = 0 \land \neg P(i,0)
           \end{cases}
           \)
         \item Kante 8: -
         \item Kante 2: -
         \item Kante 10: -
         \item Kante 12: -
      \end{enumerate}
      }

      \item {Ordnung: \(i^\prime < i, j^\prime\) beliebig
          \begin{enumerate}
            \item Kante 9,13,7,4: -
            \item Kante 6: \\
              wegen Optimierung: \(j= 0 \land \neg P^\prime (i,j)\) \\
              Ordnung: \(i^\prime < i \) \\
              Konflikt: \( i^\prime + j^\prime = i + 0 \) \\
              Existenz: \( P^\prime(i^\prime, j^\prime) \) \\
              neue Quellen: \( \langle ( i^\prime, i-i^\prime ), 6 \rangle \text{ für } j=0, i^\prime < i , P(i^\prime, i - i ^\prime) \land \neg P(T, i-T) \text{ für } i^\prime < T \leq i \) \\

\( src^{(3)}_{n1} =
\begin{cases}
  \langle (i,j), 6 \rangle, & \mbox{ für } P(i,j) \\
  \langle (i,j-1), 3 \rangle, & \mbox{ für } j \geq 1 \land \neg P (i,j) \\
  \langle (i^\prime, i-i^\prime),6 \rangle, & \mbox{ für } j=0 \land i^\prime < i \land P(i^\prime, i - i^\prime) \land \neg P(T,i-T) \\ & \mbox{ für } i^\prime < T \leq i \land i^\prime \geq 0 \land i \geq 1 \\
  \perp, & \mbox{ für } j=0 \land \neg P(i^\prime, i - i^\prime) \text{ für alle } 0 \leq i^\prime \leq i
\end{cases} \)

\item Kante 3:\\
  wegen Optimierung: \( i^\prime = i - 1, j=0 \stackrel{Konfl. Gl.}{\Rightarrow} j^\prime = 0\) \\
  neue Quelle: \( \langle (i-1,0), 3 \rangle \text{für} j=0 \land i \geq 1 \land \neg P(i,0) \land \neg P(i-1,1) \) \\
\( \Rightarrow src =
\begin{cases}
  \langle (i, j), 6 \rangle, & \mbox{für } P(i,j) \\
  \langle (i, j-1), 3 \rangle, & \mbox{für } j \geq 1 \land \neg P(i,j) \\
  \langle (i-1, 1), 6 \rangle, & \mbox{für } j=0 \land \neg P(i,j) \land P(i-1,1) \\
  \langle (i-1, 0), 3 \rangle, & \mbox{für } j=0 \land \neg P(i,j) \land \neg P(i-1,1) \land i \geq 1 \\
  \perp, & \mbox{für } i=j=0 \land \neg P(0,0)
\end{cases}
\)
\item Kanten 8, 2, 10, 1: -
           \end{enumerate}
           }

\end{enumerate}
Da die Quelle für \( (0,0) \) nicht gefunden wird, kann nicht vorab abgebrochen werden und der Algorithmus terminiert erst mit dem vollständigen Durchlauf.
