%!TEX root = Zusammenfassung.tex

\section{Normalisierung der Array-Indizes}

Einfache Abhängigkeitsanalyse macht keine symbolische Auswertung,
insbesondere setzt sie Skalarvariablen, die als Arrayindizes auftreten,
nicht ein, obwohl man deren Wert manchmal bestimmen kann. Beispielsweise
wäre im folgenden Programm jede Instanz der Array-Zuweisung von jeder
anderen ausgabeabhängig, weil der Arrayindex $x$ als unbekannt
angenommen wird:

% {\sf\begin{tabbing}
% \ins\=\ins\=\ins\=\ins\=\ins\=\kill
%   for $i := 1$ to $n$ do\\
%   \> $x := 2*i+4$;\\
%   \> $A[x] := ...$;\\
%   enddo
% \end{tabbing}}

\begin{procedure}
\SetAlgoLined
\For{$i:= 1$ \KwTo $n$}{
     $x := 2 \cdot i+4$;\\
     $A[x] := ...$;\\
}
\end{procedure}

\subsection{Skalar-Vorwärts-Ersetzung}
\label{sec:sve}

Direktes Einsetzen der Definition des Skalars in den Arrayindex liefert
das linke Programm. ACHTUNG: Gültigkeitsbereiche beachten! Im rechten
Programm ist das Ersetzen des Arrayindexes $j$ durch seine Definition
$i+1$ \emph{verboten} (vgl. \cite{Zima90}, S.~178 f.):\\[1cm]
\begin{minipage}{.4\textwidth}
    \begin{algorithm}[H]
        \For{\( i:= 1 \) \KwTo \( n \)}{
               $x := 2 \cdot i+4$;\\
               $A[2 \cdot i+4] := ...$;\\
        }
    \end{algorithm}
\end{minipage}
\begin{minipage}{.4\textwidth}
    \begin{algorithm}[H]
    $j := i+1$;\\
    $i := 0$;\\
    $A[j] := ...$;\\
    \end{algorithm}
\end{minipage}

\paragraph{Beispiel}

\begin{procedure}[H]
\SetAlgoLined
\For{$i=..$ \KwTo $n$}{
	$x=2i+4$ \\
		 ...  = \(A[x]\)  \\
}
\end{procedure}
Abhängigkeit: $i \rightarrow i+1$ schon wegen $x$

$\leadsto$

\begin{procedure}[H]
\SetAlgoLined
\For{$i=1$ \KwTo $1000$}{
	 	   ... = \(A[2i+4]\)
}
\end{procedure}

$\Rightarrow$ keine Abhängigkeit


\subsection{Wrap-Around-Variablen-Ersetzung}
\label{sec:wave}

Ähnlich zu Fall \ref{sec:sve}; allerdings ist die Skalar-Definitioin
textuell nach dem Arrayzugriff.

Das Problem ist einfach mit \emph{loop rerolling} zu lösen, wenn die
Initialisierung ``zum Schleifenprogramm paßt''; im allgemeinen ist aber
\emph{loop peeling} nötig, um Unregelmäßigkeiten an den Schleifengrenzen
zu beseitigen. Den Hauptanwendungsbereich für diese Transformation
bilden zyklische Arrays (daher wrap-around; vgl. \cite{Zima90}, S.~183).

Beispiel: Wenn $c=2$ ist, dann ist das linke Programm  äquivalent zu dem
mittleren; ansonsten ist es nur äquivalent zu dem rechten Programm:\\

\begin{minipage}[c]{4cm}
\begin{algorithm}[H]
    \( x := c \);\\
    \For{\( i:= 0 \) \KwTo \( n \)}{
        \( A[x] := \dots \);\\
        \( x := 2 \cdot i + 4 \);
    }
\end{algorithm}
% {\sf\begin{tabbing}
% \ins\=\ins\=\ins\=\ins\=\ins\=\kill
%   $x := c$;\\
%   for $i := 0$ to $n$ do\\
%   \> $A[x] := ...$;\\
%   \> $x := 2*i+4$;\\
%   enddo\\
% \end{tabbing}}
\end{minipage}
\begin{minipage}[c]{4cm}
\begin{algorithm}[H]
    \For{\( i:= 0 \) \KwTo n}{
        \( x := 2 \cdot i + 4 - 2 \);\\
        \( A[x] := \dots \);
    }
    \( x := 2 \cdot n + 4 \);
\end{algorithm}
% {\sf\begin{tabbing}
% \ins\=\ins\=\ins\=\ins\=\ins\=\kill
%   for $i := 0$ to $n$ do\\
%   \> $x := 2*i+4-2$;\\
%   \> $A[x] := ...$;\\
%   enddo\\
%   $x := 2*n+4$;\\
% \end{tabbing}}
\end{minipage}
\begin{minipage}[c]{4cm}
    \begin{algorithm}[H]
        \( A[c] := \dots \);\\
        \For{\( i := 1 \) \KwTo \( n \)}{
            \( x := 2 \cdot i + 4 - 2 \);\\
            \( A[X] := \dots \);
        }
        \( x := 2 \cdot n + 4 \);
    \end{algorithm}
% {\sf\begin{tabbing}
% \ins\=\ins\=\ins\=\ins\=\ins\=\kill
%   $A[c] := ...$;\\
%   for $i := 1$ to $n$ do\\
%   \> $x := 2*i+4-2$;\\
%   \> $A[x] := ...$;\\
%   enddo\\
%   $x := 2*n+4$;
% \end{tabbing}}
\end{minipage}


\subsection{Induktions-Variablen-Ersetzung}
\label{sec:ive}

Einfache Induktionsvariablen sind Variablen, denen nur durch (ggfs.
mehrere) Terme $v := v + c_0$ Werte zugewiesen werden; allgemeine
Induktionsvariablen sind Variablen, denen einmalig mit einem Term der
Form $v':= c_1*v + c_2$ ein Wert zugewiesen wird, wobei $v$ eine
einfache Induktionsvariable sein muß.  Sie können stets durch einen
affinen Ausdruck in den Schleifenindizes
ersetzt werden:\\

\begin{minipage}[c]{4cm}
    \begin{algorithm}[H]
        \( x:= 2 \);\\
        \For{\( i:=0 \) \KwTo \( n \)}{
            \( x := x+2 \);\\
            \( A[x] := \dots \);
        }
    \end{algorithm}
% {\sf\begin{tabbing}
% \ins\=\ins\=\ins\=\ins\=\ins\=\kill
%   $x := 2$;\\
%   for $i := 0$ to $n$ do\\
%   \> $x := x+2$;\\
%   \> $A[x] := ...$;\\
%   enddo
% \end{tabbing}}
\end{minipage}
%
\begin{minipage}[c]{4cm}
\centerline{ist äquivalent zu}
\end{minipage}
%
\begin{minipage}[c]{4cm}
    \begin{algorithm}[H]
        \For{\( i:= 0 \) \KwTo \( n \)}{
            \( x:= 2 \cdot i + 4 \);\\
            \( A[x] := \dots \);
        }
    \end{algorithm}
% {\sf\begin{tabbing}
% \ins\=\ins\=\ins\=\ins\=\ins\=\kill
%   for $i := 0$ to $n$ do\\
%   \> $x := 2*i+4$;\\
%   \> $A[x] := ...$;\\
%   enddo
% \end{tabbing}}
\end{minipage}

\bigskip
Anschließend an \ref{sec:wave} und \ref{sec:ive} kann man \ref{sec:sve}
anwenden, und schließlich kann man das entstandene Programm mit Hilfe von
\emph{Dead-Code-Eliminierung} noch vereinfachen.