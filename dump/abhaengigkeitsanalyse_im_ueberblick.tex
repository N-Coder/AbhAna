\setcounter{section}{3}
\section{Abhängigkeitsanalyse im Überblick}


\subsection{Abhängigkeiten im allgemeinen}

Abhängigkeiten in einem Programm spiegeln die durch das Programm
zwingend vorgeschriebene Ausführungsreihenfolge von \emph{Instanzen von
Anweisungen} wider. In imperativen Programmen erfolgt ihre Berechnung
üblicherweise durch die Ermittlung von Speicherstellen, auf die mehrfach
zugegriffen wird. 


\paragraph{Bedingungen}

Eine Operation $o_t$ ist abhängig von einer Operation $o_s$, gdw.
folgende Bedingungen erfüllt sind:
\begin{enumerate}
\item $o_s$ und $o_t$ müssen auf dieselbe
  Speicherzelle zugreifen (\emph{Speicherkonflikt}).
\item $o_s$ und $o_t$ müssen beide durch
  das Programm wirklich generiert werden (\emph{Existenz}).
\item $o_s$ wird gem. der sequentiellen Ausführungsreihenfolge vor $o_t$ 
  ausgeführt---es kann nur eine später ausgeführte Instanz von einer früher
  ausgeführten Instanz abhängig sein (\emph{Ordnung}).
\item Zwischen den zwei Instanzen wird der Wert der Speicherzelle nicht
  mehr modifiziert, d.h., es besteht eine ``direkte Abhängigkeit''
  (\emph{Optimierung}).
\item Mindestens einer der in Konflikt stehenden Zugriffe muß schreibend
  sein (\emph{Bernsteinzusatz}).
\end{enumerate}

\paragraph{Bemerkungen:} Die Optimierung wird bei der praktischen
Berechnung oft vernachlässigt. 

%\paragraph{Bemerkung:} 
Die Bernstein-Bedingungen werden i.a. explizit angegeben (3 Fälle) und
beinhalten Ordnung, Existenz und Speicherkonflikt implizit. Je nach
Analyse-Methode wird die Zusatzbedingung von Bernstein implizit bei der
Auswahl der zu betrachtenden Zugriffspaare berücksichtigt.


\subsection{Abhängigkeiten in Schleifenprogrammen}

Die Abhängigkeits-Analysetechniken sind für verschachtelte
Schleifenprogramme mit Arrays als einziger Datenstruktur relativ gut
entwickelt (Skalar-Variable können dabei als null-dimensionale Arrays
interpretiert werden). 

\noindent
Damit ergibt sich:

\begin{enumerate}
\item Instanzen von Anweisungen sind durch die Anweisung selbst und den
  zugehörigen Iterationsvektor (Vektor der Werte aller umgebenden
  Schleifenindizes) eindeutig charakterisiert.
\item Der Indexraum einer Anweisung ist die Menge aller durch das
  Schleifenprogramm aufgezählten Indexvektoren für diese Anweisung.
\item Die sequentielle Ausführungsreihenfolge von zwei Instanzen ist
  gegeben durch die lexikographische Ordnung auf dem gemeinsamen Präfix
  ihrer Iterationsvektoren und---bei deren Gleichheit---durch die
  textuelle Reihenfolge der Anweisungen im Programmtext.
\item Speicherkonflikt liegt genau dann vor, wenn zwei Instanzen auf
  dasselbe Array zugreifen und für jede Array-Dimension auch denselben
  Index liefern. ACHTUNG: Nicht die Schleifen-Indizes, sondern die
  Array-Indizes nach dem Einsetzen der Schleifen-Indizes
  (=Iterationsvektoren) müssen übereinstimmen.
\end{enumerate}

\paragraph{Beispiel:} 3 Schleifen, 2-dim. Arrays.

\paragraph{Begriffe:}

\begin{enumerate}
\item Die Abhängigkeit geht von der \emph{Quelle} zum \emph{Ziel}.
\item Im Fall von perfekt verschachtelten Schleifen ist der
  \emph{Abhängigkeitsvektor} (oder \emph{Distanzvektor}) ist die Differenz: Ziel-Iterationsvektor minus
  Quell-Iterationsvektor. Er ist also immer lexikographisch nicht-negativ (also hinsichtlich des zeitlichen Ablaufs).
\item Eine Abhängigkeit heißt \emph{uniform}, wenn der Abhängigkeitsvektor an
  jedem Punkt des Indexraumes identisch ist.
\item Durch \emph{Richtungsvektoren}, in denen die Werte der
  Abhängigkeitsvektoren durch ihr Signum ersetzt werden, kann man selbst
  nicht-uniforme Abhängigkeiten (zu Lasten der Genauigkeit) effizient
  darstellen. 
\item Eine Abhängigkeit wird von derjenigen Verschachtelungstiefe
  \emph{getragen}, deren erster Eintrag im Richtungsvektor größer als
  null ist. Ist der Richtungsvektor gleich dem Nullvektor, so heißt die
  Abhängigket \emph{schleifenunabhängig}.
\end{enumerate}


Je nachdem, ob das Ziel oder die Quelle ein lesender oder ein
schreibender Zugriff ist, werden verschiedene \emph{Typen} von
Abhängigkeiten unterschieden: 

\smallskip

\begin{center}
\begin{tabular}{||r|cc||}
\hline
Ziel & \multicolumn{2}{|c||}{Quelle}\\
 & liest & schreibt \\
\hline
liest & input & true\\
schreibt & anti & output\\
\hline
\end{tabular}
\end{center}

\paragraph{Bemerkungen:} 

\begin{enumerate}
\item I.A. werden input dependences wegen der Bernstein-Zusatzbedinung
  vernachlässigt.
\item True dependences, die das Optimierungskriterium erfüllen, heißen
  auch flow dependences. (ACHTUNG: Diese Begriffe werden in der Literatur
  uneinheitlich verwendet.)
\end{enumerate}

\paragraph{Graphische Darstellung:}

Die graphische Darstellung erfolgt entweder durch den
\emph{Iterations-Abhängigkeitsgraphen} oder den
\emph{Statement-Abhängigkeitsgraphen}.
