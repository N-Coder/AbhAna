\begin{titlepage}

\centerline{\huge DISCLAIMER}

\vspace*{1cm}

\large
Die zur Vorlesung erscheinenden Unterlagen sind nicht als vollständiges
Skript gedacht. Sie alleine reichen auch nicht für eine
Prüfungsvorbereitung. Gründe:
\begin{itemize}
\item Sie sind absolut informell und daher mißverständlich.
\item Sie sind unvollständig.
\item Sie enthalten keinerlei Beispiele.
\item Sie sind ohne Vorlesung vermutlich unverständlich.
\end{itemize}

\vspace*{1cm}
\textbf{Was sollen diese Unterlagen dann überhaupt?}

\vspace*{5mm}

Sie sollen dazu dienen, den Inhalt zu strukturieren und das Aufschreiben
von oft nur gesprochenen Erklärungen zum Teil unnötig werden zu lassen,
damit man während der Vorlesung leichter mitdenken kann. Die starke,
numerierte Gliederung der Unterlagen er\-möglicht (hoffentlich) die
einfache Ergänzung der Unterlagen durch eine eigene Mitschrift, die dann 
etwa die Formalisierungen und die Beispiele von der Tafel beinhalten
könnte.

\vspace*{5mm}

Für die Prüfungsvorbereitung kann sie als Leitfaden dienen, der zwar
\textbf{voraussichtlich} alle wesentlichen Themengebiete der Vorlesung
beinhaltet, sie jedoch meist nicht in der nötigen Tiefe behandelt.

\vspace*{1cm}

Wenn die Vorbereitungszeit zu knapp wird, dann behalte ich mir natürlich 
vor, die Unterlagen für einzelne Vorlesungen nicht auszuarbeiten (wobei
ich hoffe, daß das nicht der Fall sein wird).

\vspace*{1cm}

\small
Trotz aller Schwächen bleibt das Copyright bei Martin Griebl.
\end{titlepage}
