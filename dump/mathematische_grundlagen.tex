\setcounter{section}{2}
\section{Mathematische Grundlagen}
\label{sec:math:i}

\begin{enumerate}
\item unimodulare Abbildungen 
  \begin{enumerate}
  \item ganzzahlig; Determinante=$\pm$1; Inverses ganzzahlig
  \item Bilder von Polytopen unter unimodularen Abbildungen
  \item elementare Zeilentransformationen sind unimodulare Abbildungen:
    \begin{enumerate}
    \item Reversal: Multiplikation einer Zeile mit -1 (!)
    \item Interchange: Vertauschen zweier Zeilen
    \item Skewing: ein Vielfaches einer Zeile zu einer anderen addieren
    \end{enumerate}
  \item Unimodularität abgeschlossen unter Multiplikation
  \item Wesentliches Verfahren: Zeilen-Stufen-Reduktion (Echelon
    Reduktion) einer Matrix $A$ durch eine unimodulare Transformation
    $U$: $U*A = S$
  \item Möglichkeit: Diagonalisierung durch anschließende elementare
    Spaltenoperationen $V$: $S*V = D$
%  \item keine Permutationsmatrizen
  \end{enumerate}
\item ganzzahlig-lineare Gleichungssysteme
  \begin{enumerate}
  \item ``normale'' Gauß-Elimination liefert alle rationalen Lösungen
  \item Einsetzen der frei wählbaren Variablen führt i.A. zu Brüchen in
    den nicht frei wählbaren Variablen
  \item Nachbearbeitung der freien Variablen ist möglich (um
    ganzzahlige Lösungen zu erhalten). Frage: gibt es eine direktere
    Berechnung?  
  \item Bekannt: eine ganzzahlig-lineare Gleichung ist lösbar gdw. der
    ggT $g$ der Koeffizienten $a$ die rechte Seite $c$ teilt
  \item ggT$(a)$ steht in der Zeilen-Stufen-Form des Spaltenvektors
    $(a)$ oben; erste Zeile von $U$ liefert die Multiplikatoren für $a$
  \item Idee: ggT-Restriktion als erstes einsetzen um die Brüche zu
    vermeiden, die bei der Rückwärtssubstitution \`a la Gauß bei der
    letzten Substitution entstehen. Löungsmöglichkeiten:
    Spalten-Stufen-Form (ungewohnt) oder ``transponierte
    Darstellung''. Daher ab sofort:
  \item Variablenvektor $x$ von links multiplizieren (damit Matrizen
    transponiert) und Zeilen-Stufen-Transfor\-mation.  Also: $x*A =
    c$. Dann gilt für eine einzelne Gleichung $x*a=c$ nach unimodularer
    Zeilen-Stufen-Transforma\-tion des Spaltenvektors $a$:
  \item Menge aller Lösungen: $(x_1,\cdots,x_m) = (c/g,t_2,\cdots,t_m)*
    U$
  \item Verallgemeinerung: $x*A = c$ lösbar gdw. $\exists t: t*S =
    c$. Menge aller Lösungen: $x = t*U$.
%  \item keine 2 Variables mit \psi_- und \psi_+
  \end{enumerate}

% sd: Beispiel
\textbf{Beispiel}\\
Ziel: Eine ganzzahlige Lösung ist gewünscht \\
Erinnerung: $U*A = S$ und $x = t*U$

Gleichungssystem:

\begin{align*}
 4x_1 +  6x_2        &= 8 \\
15x_1 + 21x_2 + 6x_3 &= 9
\end{align*}

Lösung nach Algorithmus:

$\left( 
\begin{array}{cc|ccc}
4 & 15 & 1 & 0 & 0\\ 
6 & 21 & 0 & 1 & 0\\ 
0 & 6  & 0 & 0 & 1%
\end{array}%
\right) \leadsto \left( 
\begin{array}{cc|ccc}
2 & 6 & -1 &  1 & 0\\ 
0 & 3 &  3 & -2 & 0\\ 
0 & 0 & -6 &  4 & 1%
\end{array}%
\right) $

Somit:

$\left(
t_1, t_2, t_3
\right) *
\left(
\begin{array}{cc}
2 & 6 \\
0 & 3 \\
0 & 0%
\end{array}
\right)
= 
\left(
\begin{array}{cc}
8 & 9
\end{array}
\right)$

$\Rightarrow t_1 = 4$, $t_2 = \frac{(9-6*4)}{3} = -5$, $t_3 \in \mathbb{Z}$


$\left(
\begin{array}{c|cc}
4 & 1 & 0\\
6 & 0 & 1%
\end{array}
\right) \leadsto \left(
\begin{array}{c|cc}
2 & -1 & 1\\
0 &  3 & -2%
\end{array}
\right)
$
\\
\\Damit ergibt sich als ggT die 2 und als Lösung des Ursprungssystems:

$x = (x_1, x_2, x_3) = t * U = 
\\
(4,-5,t_3) *
\left(
\begin{array}{ccc}
-1 & 1 & 0 \\
3 & -2 & 0 \\
-6 & 4 & 1
\end{array}
\right)
=
(-19-6t_3, 14+4t_3, t_3)
$

% /sd
\item lineare Ungleichungssysteme und Fourier-Elimination
  sukzessive Elimination der einzelnen Variablen: sei $x_j$ die zu
  eliminierende Variable
  \begin{enumerate} 
  \item sortiere die Ungleichungen in untere Schranken für $x_j$, obere
    Schranken für $x_j$ und Ungleichungen, die $x_j$ nicht beinhalten
  \item normiere die beschränkenden Ungleichungen (Koeffizienten der
    zu eliminierenden Variablen alle gleich eins)
  \item lösche die $x_j$ beschränkenden Ungleichungen und füge
    stattdessen die Paare aller möglichen Ungleichungskombinationen ein, 
    die sich ergibt, wenn man alle Unterschranken von $x_j$ allen
    Oberschranken von $x_j$ gegenüberstellt
  \item das resultierende System hat u.U. wesentlich mehr Ungleichungen, 
    aber eine Variable weniger; also: eliminiere nächste Variable
  \item erfolgreiche Termination: keine Ungleichung oder keine Variable
    geblieben (keine Ungleichung = unbeschränkte Variable)
  \item erfolglose Termination: die letzte Ungleichung (ohne
    Variable!!) ist nicht erfüllt
  \end{enumerate}

\textbf{Beispiel}\\
Lösung mit Fourier-Motskin-Algorithmus:
\begin{align*}
0 &\leq t-p\\
t-p &\leq n\\
p &\geq 0\\
p &\leq t-p+2
\end{align*}

\newpage

Somit:
\begin{enumerate}
    \item innerste Schleife\\
$\left.
    \begin{array}{ccc}
        p    &\leq t &\leq n+p \\
        2p-2 &\leq t &\leq n+p
    \end{array}
\right\}$ t unabhängig; Schleifenrumpf:
\begin{procedure}[ht]
\For{$(t=\lceil max(p,2p-2) \rceil ; t \leq \lfloor min(n+p,n+p) \rfloor ; t++)$}{}
\end{procedure}


    \item t eliminiert\\
$\left.
0 \leq p \leq n+2
\right\}$ erweiterter Schleifenrumpf:
\begin{procedure}[ht]
\For{$(p=0; p \leq n+2; p++)$} {
    \For{$(t=\lceil max(p,2p-2) \rceil ; t \leq \lfloor min(n+p,n+p) \rfloor ; t++)$}{}
}
\end{procedure}

    \item p eliminiert\\
$\left. 0 \leq n+2 \right\}$ lösbar, aber nur im Rationalen.
\end{enumerate}


\item affine anstatt linearer Transformationen:
  \begin{enumerate}
    \item Strukturparameter werden wie zusätzliche Variablen behandelt,
      die ``zufällig'' nur einen einzigen Wert zur Laufzeit annehmen.

      Um sicherzustellen, daß sich ihr Wert nicht ändert, werden in den
      Transformationsmatrizen Zeilen eingefügt, die jeden
      Strukturparameter auf sich selbst abbilden.
    \item Eine additive Konstante wird wie ein zusätzlicher
      Strukturparameter behandelt, der zufällig nur den Wert 1 annimmt.
    \item Mathematischer Hintergrund: homogene Koordinaten.
      
      Der Vektor $(x_1,\cdots,x_n,\lambda)$ in homogenen Koordinaten
      entspricht für $\lambda\not=0$ dem Vektor $({x_1\over
        \lambda},\cdots,{x_n\over \lambda})$ in den üblichen
      kartesischen Koordinaten. Für $\lambda=0$ entspricht der Vektor
      $(x_1,\cdots,x_n,\lambda)$ einem Punkt im Unendlichen in Richtung
      $(x_1,\cdots,x_n)$.
  \end{enumerate}
\end{enumerate}
