\setcounter{section}{1}
\section{Polyedermodell}
\label{sec:polymod}


\subsection{Das ursprüngliche Modell}
\label{sec:orig-mod}

\begin{enumerate}
\item Restriktionen:
\begin{enumerate}
\item perfekt verschachtelt
\item Schleifengrenzen: affine Ausdrücke in Strukturparametern und umgebenden
  Indizes und Maxima bzw. Minima davon (für Unter- bzw. Obergrenzen)
\item Arrayindizes affin in den Schleifenindizes und Strukturparametern
\item uniforme Abhängigkeiten
\item ein Schedule und eine Allokation für den kompletten Rumpf;
  Schedule zunächst eindimensional
%\item Schedule und Allokation linear, voneinander lin. unabhängig,
%  gemeinsam volle Dinensionalität
\end{enumerate}
%
\item Modell:
\begin{enumerate}
\item $n$-dimensionaler Schleifensatz im $n$-dimensionalen Raum (je
  Schleife eine Dimension)
\item jeder affine Ausdruck in den Schleifengrenzen definiert einen
  Halbraum
\item Polyeder: der Durchschnitt endlich vieler Halbräume
\item Polytop: beschränktes Polyeder 
\item (Quell-)Indexraum ist ein Polytop
\item Abhängigkeiten durch Pfeile im Indexraum repräsentiert
\item Quellpolytop beinhaltet alle nötigen Informationen
\end{enumerate}
%
\item Raum-Zeit-Abbildung:
\begin{enumerate}
\item Schedule: Funktion, die jeder Operation einen (logischen)
  Ausführungszeitpunkt zuordnet, und dabei die durch die Abhängigkeiten
  vorgegebenen Bedingungen berücksichtigt
\item für das Modell: affine Funktion
\item graphische Bestimmung s. Beispiel der Vorlesung
\item mathematische Bestimmung: lineare Programmierung. Aufgabe:
  minimiere die affine Funktion unter der Nebenbedingung, daß ihr Wert
  für einen abhängigen Punkt größer ist als ihr Wert an dem Punkt, von
  dem er abhängt.
\item Allokation: Funktion, die jeder Operation einen (virtuellen)
  Prozessor zur Ausführung zuordnet
\item für das Modell: affine Funktion, die zum Schedule linear
  unabhängig ist (wegen der realen Maschinen und wegen des Modells
  nötig!)
\item Raum-Zeit-Abbildung: die (mehrdimensionale) affine Abbildung,
  repräsentiert durch die Transformationsmatrix, die sich aus Schedule
  und Allokation zusammensetzt, und so jeder Operation einen
  Raum-Zeit-Punkt der Ausführung zuweist. Zusätzliche Restriktion an die 
  Allokation: die Raum-Zeit-Matrix muß unimodular sein (Determinante
  betragsmäßig 1)
\end{enumerate}
%
\item Transformation des Modells:
\begin{enumerate}
\item Problem: jede einzelne Dimension kann teilweise in Raum und
  teilweise in der Zeit sein
\item Ziel: Separation -- jede Dimension entweder im Raum oder in der Zeit
\item Weg: Basiswechsel = Anwendung der Raum-Zeit-Abbildung auf den
  Quell-Indexraum
\item Resultat: ``verzerrtes'' Polytop: Zielpolytop
\end{enumerate}
%
\item Zurück zum Programm:
\begin{enumerate}
\item Aufgabe: ``scanning'' der Punkte des Zielpolytops
\item parallele Schleifen für die Raumdimensionen und sequentielle
  Schleifen für die Zeitdimensionen
\item Weg: im Quell-Ungleichungssystem die Indizes durch
  ``$T^{-1}*$Zielindizes'' ersetzen und auflösen
\item abhängig von der gewünschten Reihenfolge der Zieldimensionen:
  (a)synchrones Programm
\item Quellindizes durch ``$T^{-1}*$Zielindizes'' ersetzen
\end{enumerate}
\end{enumerate}

\subsection{Das erweiterte Modell}
\label{sec:new-mod}

Erweiterungen:
\begin{enumerate}
\item affine Abhängigkeiten\\[-7mm]
\item stückweise affine Schedules\\[-7mm]
\item per-Statement-Schedule\\[-7mm]
\item nicht-perfekte Verschachtelung\\[-7mm]
\item nicht-unimodulare Raum-Zeit-Abbildungen\\[-7mm]
\item beliebige (z.B. nicht-invertierbare) Raum-Zeit-Abbildungen\\[-7mm]
\item beliebige Schleifentypen\\[-7mm]
\item allgemeine Array-Indizes\\[-7mm]
\end{enumerate}
