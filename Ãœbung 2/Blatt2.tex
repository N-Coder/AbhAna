\documentclass[a4paper,10pt]{article}

% IMPORTS
\usepackage{amsfonts}
\usepackage{amsmath}
\usepackage{amssymb}
\usepackage{graphicx}
\usepackage{titlesec}
\usepackage{wrapfig}
\usepackage[ngerman]{babel}
\usepackage[utf8x]{inputenc}
\usepackage{pdfpages}
\usepackage{placeins}
\usepackage{color}
\usepackage{eurosym}
\usepackage{xargs}
\usepackage{xcolor}
\usepackage{subcaption}
\usepackage{hyperref}
\usepackage[margin=2cm]{geometry}
\usepackage[colorinlistoftodos,prependcaption,textsize=tiny]{todonotes}

% CONFIG
\clubpenalty = 9000
\widowpenalty = 9000
\displaywidowpenalty = 9000
\titlespacing\section{0pt}{14pt plus 4pt minus 2pt}{2pt plus 2pt minus 1pt}
\titlespacing\subsection{0pt}{10pt plus 4pt minus 2pt}{2pt plus 2pt minus 1pt}
\setlength{\parindent}{0pt}
\setlength{\parskip}{0.5em}
\setcounter{tocdepth}{2}
\newcommand{\cell}[2][c]{\begin{tabular}[#1]{@{}l@{}}#2\end{tabular}}
\renewcommand*\arraystretch{1.3}

% -----------------------------------------------------------------------------
\begin{document}

\begin{flushright}
    Niko Fink\\
    Fabian Knorr
\end{flushright}
\vspace*{-5.5em}

\section*{Aufgabe 3}
\subsection*{Aufgabe 3a)}

\begin{itemize}
    \item Hintereinanderausführungen erzeugen einen Konkatenationsoperator $(;)$ mit
        erstem Statement als linken, zweitem Statement als rechten Kindknoten
    \item \texttt{if}-Statements erzeugen einen if-Knoten mit drei Kindknoten:
        der Bedingung, dem then-Zweig (links) und dem else-Zweig (rechts).
    \item \texttt{do}-Schleifen erzeugen einen do-Knoten mit Schleifenkörper und
        Bedingung als Kindknoten.
\end{itemize}

\begin{center}
    \includegraphics[scale=0.7]{023a.pdf}
\end{center}

\newpage
\subsection*{Aufgabe 3b)}

\begin{itemize}
    \item Beginne bei den Blättern. Überschriebene Ausdrücke landen in der kill-Menge,
        Geschriebene in der gen-Menge.
    \item Bei einer Konkatenation ist die gen-Menge gen-Menge des zweiten Kindknotens
        und die Ausdrücke der gen-Menge des ersten Knotens, die nicht in der kill-Menge
        des zweiten liegen.
    \item Bei einem do-Knoten sind gen- und kill-Mengen die des Schleifenkörpers.
    \item Bei einem if-Knoten ist die gen-Menge Vereinigung der gen-Mengen der Kinder,
        die kill-Menge Schnitt der kill-Mengen
\end{itemize}

\begin{center}
\includegraphics[scale=0.55]{023b.pdf}
\end{center}

\newpage
\subsection*{Aufgabe 3c)}

\begin{itemize}
    \item Beginne bei der Wurzel. Die in-Menge ist $\emptyset$.
    \item Bei einer Konkatenation ist die in-Menge des ersten Kindes die des Elternknotens,
        die des zweiten die out-Menge des ersten und die out-Menge des Elternknotens die des
        zweiten Kindes.
    \item Bei einem Blattknoten ist $out = gen \cup (in \setminus kill)$.
    \item Bei einem if-Knoten ist die out-Menge Vereinigung der out-Mengen der Kinder.
    \item Bei einem do-Knoten ist die out-Menge die des Schleifenkörpers.
\end{itemize}


\begin{center}
\includegraphics[scale=0.55]{023c.pdf}
\end{center}

\subsection*{Aufgabe 3c)}

...

\newpage
\section*{Aufgabe 4}

\textbf{Liveness-Analyse:} Es soll ermittelt werden, wie lange ein Wert gebraucht wird (z.B. zur
Registerallokation). Interessant ist also die Zeit vom Schreib- bis zum letzten Lesevorgang.

Vorgehen \textit{bottom-up}: Lesezugriffe erzeugen die gen-Menge, Schreibzugriffe die kill-Menge.

\vspace*{1em}
\begin{center}
    \begin{tabular}{cl}
        $S: \hdots := \hdots$ & \cell{%
            $gen[S] = vars_{read}$\\
            $kill[S] = vars_{written} \setminus gen[S]$ \\
            $in[S] = gen[S]\cup(out[S]\setminus kill[S])$}\\
        \hline
        $S: S1; S2$ & \cell{%
            $gen[S] = (gen[S2]\setminus kill[S1]) \cup gen[S1]$\\
            $kill[S] = (kill[S2]\setminus gen[S1]) \cup kill[S1]$\\
            $in[S] = in[S1]$\\
            $out[S1] = in[S2]$\\
            $out[S2] = out[S]$}\\
        \hline
        $S: \mathrm{if} \hdots S1\;\mathrm{else}\;S2$ & \cell{%
            $gen[S]=gen[S1]\cup gen[S2]$\\
            $kill[S]=kill[S1]\cap kill[S2]$\\
            $in[S] = in[S1]\cup in[S2]$\\
            $out[S1] = out[S]$\\
            $out[S2] = out[S]$}\\
        \hline
        $S: \mathrm{do}\;S1\;\mathrm{while}\hdots$ & \cell{%
            $gen[S] = gen[S1]$\\
            $kill[S]=kill[S1]$\\
            $in[S] = in[S1]$\\
            $out[S1] = out[S]\cup gen[S1]$}\\
    \end{tabular}
\end{center}

\end{document}
