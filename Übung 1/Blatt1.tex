\documentclass[a4paper,10pt]{article}

% IMPORTS
\usepackage{amsfonts}
\usepackage{amsmath}
\usepackage{amssymb}
\usepackage{graphicx}
\usepackage{titlesec}
\usepackage{wrapfig}
\usepackage[ngerman]{babel}
\usepackage[utf8x]{inputenc}
\usepackage{pdfpages}
\usepackage{placeins}
\usepackage{color}
\usepackage{eurosym}
\usepackage{xargs}
\usepackage{xcolor}
\usepackage{subcaption}
\usepackage{hyperref}
\usepackage[margin=2cm]{geometry}
\usepackage[colorinlistoftodos,prependcaption,textsize=tiny]{todonotes}

% CONFIG
\clubpenalty = 9000
\widowpenalty = 9000
\displaywidowpenalty = 9000
\titlespacing\section{0pt}{14pt plus 4pt minus 2pt}{2pt plus 2pt minus 1pt}
\titlespacing\subsection{0pt}{10pt plus 4pt minus 2pt}{2pt plus 2pt minus 1pt}
\setlength{\parindent}{0pt}
\setlength{\parskip}{0.5em}
\setcounter{tocdepth}{2}

% -----------------------------------------------------------------------------
\begin{document}

\begin{flushright}
    Niko Fink\\
    Fabian Knorr
\end{flushright}
\vspace*{-5.5em}

\section*{Aufgabe 1}
\subsection*{Aufgabe 1a)}
\begin{center}
\includegraphics[scale=0.7]{Aufgabe1.pdf}
\end{center}

\subsection*{Aufgabe 1b)}

\begin{itemize}
    \item \textbf{Maximale Ketten:}
    Von der $1$ an beginnend den Baum aufbauen, Knoten streichen wenn sie durch einen längeren
    Pfad erreicht werden können. Tritt ein Knoten mehrfach mit gleicher Distanz zum Anfangsknoten
    auf, bleiben beide Ketten bestehen. Sind alle Kanten ausgeschöpft, mit einem anderen Knoten von
    vorne beginnen, der noch in keinem Baum aufgetaucht ist (Im Beispiel nicht der Fall).

    Hier:
\[1 \rightarrow 3 \rightarrow 6 \rightarrow 9\]
\[1 \rightarrow 3 \rightarrow 6 \rightarrow 8 \rightarrow 10\]
\[1 \rightarrow 2 \rightarrow 5 \rightarrow 8 \rightarrow 10\]
\[1 \rightarrow 2 \rightarrow 4 \rightarrow 7 \rightarrow 10\]

\item \textbf{Maximale Antiketten:}
    Zunächst die transitive Hülle der Relation bestimmen. Das geschieht im Beispiel idealerweise
    bei der 10 beginnend, da Kanten nur von kleineren zu größeren Zahlen existieren. Daraus können
    die Antiketten der Länge 2 bestimmt werden. Diese lassen sich immer dann zu längeren Antiketten
    verbinden, wenn $\{a,b\}, \{a,c\}, \{b, c\}$ Antiketten sind: Sie bilden zusammen $\{a,b,c\}$.
    \textit{Maximale Antiketten} sind die mit der Länge $N$, wenn die längste Antikette $N$
    Elemente hat.

    Hier:
\[\{1\}, \{2,3\}, \{2,6\}, \{2,9\}, \{3, 4,5\}, \{3, 5, 7\}, \{4, 5, 6\}, \{4,5,9\}, \{4,8,9\},
\{5,6,7\},\{5,7,9\}, \{7,8,9\}, \{9,10\}\]
\end{itemize}

\newpage
\section*{Aufgabe 2}

\subsection*{Aufgabe 2a)}

$G$ ist schwach zusammenhängend, da er nicht in unzusammenhängende Teilgraphen zerfällt. Er ist 
\textit{nicht} stark zusammenhängend, da $b$ von $e$ aus nicht erreichbar ist.

\subsection*{Aufgabe 2b)}

\[\widetilde{V} = \{bcd, a, e\}\]
\[\widetilde{E} = \{(bcd, a), (a, e)\}\]

\subsection*{Aufgabe 2c)}

\begin{minipage}{0.49\textwidth}
    \begin{center}
        \includegraphics[scale=0.7]{Aufgabe2.pdf}
    \end{center}
\end{minipage}
\begin{minipage}{0.49\textwidth}
    \begin{center}
        \includegraphics[scale=0.7]{Aufgabe2b.pdf}
    \end{center}
\end{minipage}

\subsection*{Aufgabe 2d)}

Eine: $(e, b)$, $(e, c)$ oder $(e, d)$.

\end{document}
