\documentclass[a4paper,10pt]{article}

% IMPORTS
\usepackage{amsfonts}
\usepackage{amsmath}
\usepackage{amssymb}
\usepackage{graphicx}
\usepackage{titlesec}
\usepackage{wrapfig}
\usepackage[ngerman]{babel}
\usepackage[utf8x]{inputenc}
\usepackage{pdfpages}
\usepackage{placeins}
\usepackage{color}
\usepackage{eurosym}
\usepackage{xargs}
\usepackage{xcolor}
\usepackage{subcaption}
\usepackage{hyperref}
\usepackage[margin=2cm]{geometry}
\usepackage[colorinlistoftodos,prependcaption,textsize=tiny]{todonotes}

% CONFIG
\clubpenalty = 9000
\widowpenalty = 9000
\displaywidowpenalty = 9000
\titlespacing\section{0pt}{14pt plus 4pt minus 2pt}{2pt plus 2pt minus 1pt}
\titlespacing\subsection{0pt}{10pt plus 4pt minus 2pt}{2pt plus 2pt minus 1pt}
\setlength{\parindent}{0pt}
\setlength{\parskip}{0.5em}
\setcounter{tocdepth}{2}

% -----------------------------------------------------------------------------
\begin{document}

\begin{flushright}
    Niko Fink\\
    Fabian Knorr
\end{flushright}
\vspace*{-5.5em}

\section*{Aufgabe 1}
\subsection*{Aufgabe 1a)}
\begin{center}
\includegraphics[scale=0.7]{Aufgabe1.pdf}
\end{center}

\subsection*{Aufgabe 1b)}

\begin{itemize}
\item Maximale Ketten:
\[1 \rightarrow 3 \rightarrow 6 \rightarrow 9\]
\[1 \rightarrow 3 \rightarrow 6 \rightarrow 8 \rightarrow 10\]
\[1 \rightarrow 2 \rightarrow 5 \rightarrow 8 \rightarrow 10\]
\[1 \rightarrow 2 \rightarrow 4 \rightarrow 7 \rightarrow 10\]

\item Maximale Antiketten:
\[\{2,3\}, \{2,6\}, \{2,9\}, \{3, 4,5\},\{3,4,8\},\{3,7,8\}, \{3, 5, 7\},\]
\[\{4, 5, 6\}, \{4,5,9\}, \{4,8,9\},
\{5,6,7\},\{5,7,9\}, \{7,8,9\}, \{9,10\}\]
\end{itemize}

\newpage
\section*{Aufgabe 2}

\subsection*{Aufgabe 2a)}

$G$ ist schwach zusammenhängend, da er nicht in unzusammenhängende Teilgraphen zerfällt. Er ist 
\textit{nicht} stark zusammenhängend, da $b$ von $e$ aus nicht erreichbar ist.

\subsection*{Aufgabe 2b)}

\[\widetilde{V} = \{bcd, a, e\}\]
\[\widetilde{E} = \{(bcd, a), (a, e)\}\]

\subsection*{Aufgabe 2c)}

\begin{minipage}{0.49\textwidth}
    \begin{center}
        \includegraphics[scale=0.7]{Aufgabe2.pdf}
    \end{center}
\end{minipage}
\begin{minipage}{0.49\textwidth}
    \begin{center}
        \includegraphics[scale=0.7]{Aufgabe2b.pdf}
    \end{center}
\end{minipage}

\subsection*{Aufgabe 2d)}

Eine: $(e, b)$, $(e, c)$ oder $(e, d)$.

\end{document}
